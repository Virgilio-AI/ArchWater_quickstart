\begin{section}{neovim}
	The kick start for neovim is
	\begin{subsection}{vimrc bindings}

		this are the bindings that are done to navigate the configuration files. they can be used in any instance of neovim

		In normal mode use:

		open the init.vim
		
		edit vim

		\textbackslash ev

		edit the shortcuts.neovim

		edit vim shortcust

		\textbackslash evs

		edit the vims autocommands

		edit the vim autocommands

		\textbackslash eva

		edit vim skeletons

		\textbackslash evs


	\end{subsection}
	\begin{subsection}{file browsing inside neovim}
		In Normal mode use:

		using the nred tree plugin you can have a file tree on the side

		using normal mode type:

		\<Ctrl\> + b

		move with vim bindings

		to list available commands type a simple ? character

	\end{subsection}
	\begin{subsection}{using code runner for code forces or similar}
		vim code runner comes by default in water linux and the bindings you can try are:
		
		run code in a terminal

		if \<F11\>\<F11\> has been used \<F11\> has the same functionality

		\<F11\>

		run code in the terminal and get test cases to compare with correct samples

		\<F11\>\<F11\>\<F11\>

		force run code in terminal, this is usefull when \<F11\>\<F11\> has been used

		refer to \url{https://github.com/Virgilio-AI/vim-code-runner}

		for supported languages and for details

	\end{subsection}
	\begin{subsection}{using neovim to edit leetcode files and test code}
		vim leetcode comes by default in water linux refer \url{https://github.com/Virgilio-AI/leetcode-neovim.git}

		for details and configuration steps

		to enter leetcode mode you have to type in the terminal

		lc (leetcode problem number)

		"Example:

		lc 25

		then a neovim instance will pop up with a terminal

		leetcode neovim in water linux comes with python configured as the default

		in normal mode use:
		
		\<F11\>\<F11\> to execute code

		you can use \<F11\> to test code

		all the commands will be executed in the terminal

	\end{subsection}
	\begin{subsection}{skeletons in neovim}

		to put a new skeleton in neovim edit the skeletons.vim file

		in normal mode use:

		\textbackslash evS

		edit the snippet file example:

		python.snippet inside the UltiSnips folder in the same folder as the init.vim file

		there are already examples in thouse files just follow the same logic

	\end{subsection}
\end{section}

